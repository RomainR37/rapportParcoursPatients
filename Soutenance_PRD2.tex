\documentclass{beamer}

\usepackage[frenchb]{babel}
\usepackage[utf8]{inputenc}

\usetheme{Frankfurt}

\addtobeamertemplate{navigation symbols}{\usebeamerfont{footline}%
	\usebeamercolor[fg]{footline}
	\hspace{1em}
	\insertframenumber/\inserttotalframenumber}

\title[Outil de gestion de parcours-patient]{Outil de gestion de parcours-patient dans un hôpital de jour}
\subtitle{Soutenance de Projet R\&D}
\author{Romain ROUSSEAU}
\date{\today}


\begin{document}
	
\begin{frame}[plain]
	\titlepage
\end{frame}

\AtBeginSection[]
{
\begin{frame}
	
	\tableofcontents[currentsection, hideallsubsections]
	
\end{frame} 
}

\begin{frame}[plain]
\frametitle{Introduction}

Projet Recherche \& Développement sur l'amélioration d'un outil de gestion de parcours-patient pour l'AP-HP (Assistance publique - Hôpitaux de Paris).

\begin{figure}
	\includegraphics[scale=0.7]{images/LOGO_APHP}
\end{figure}

\end{frame}


\begin{frame}

\frametitle{Sommaire}

\tableofcontents

\end{frame}

\section[Contexte]{Contexte}

\begin{frame}
\frametitle{Contexte}

La gestion des patients dans les hôpitaux fait l'objet de nombreux débats.

\bigbreak

\begin{itemize}
	\item Temps d'attente trop longs pour les patients
	\item Problèmes liés aux ressources disponibles (salles, personnel, ...) 
	\item Problèmes budgétaires
\end{itemize}

Développement des cliniques ambulatoires (aussi appelées Hôpitaux de jour)

\end{frame}

\begin{frame}
\frametitle{Contexte}

L'outil va permettre de gérer un ensemble de patients nécessitant plusieurs activités de soin planifiées sur une journée.

\bigbreak

Un patient suit un "\textbf{parcours de soins}" défini par un ensemble d'activités au sein de l'hôpital.

\bigbreak

Une activité de soin est caractérisée par une durée et un ensemble de ressources. Certaines peuvent avoir des \textbf{contraintes de précédence}. 

\end{frame}


\begin{frame}
\frametitle{Contexte du projet}

\begin{block}{Suite de plusieurs projets consécutifs}
	\begin{description}
		\item[Projet SI et R\&D 2015-2016]: par 6 étudiants et Jean Coquelet, modélisation et développement des fonctionnalités de base
		\item[Projet R\&D 2016-2017]: par Guillaume Pochet, planification manuelle des activités
		\item[Projet R\&D 2017-2018]: par Yang Jing, ajout de fonctionnalités de gestion
	\end{description}
\end{block}

\end{frame}

\section[Description]{Description du projet}


\section[Gestion de projet]{Gestion de projet}



\section{Mise en œuvre}


\section{Qualité de code}




\section{Tests}



\section*{Conclusion}

\begin{frame}
\frametitle{Conclusion}


\end{frame}

\end{document}