\documentclass{polytech/polytech}

\typereport{prddi5}

\reportyear{2018-2019}
\title{Outil de gestion de parcours patient dans un hôpital de jour}
\student[di5]{Romain}{ROUSSEAU}{romain.rousseau@etu.univ-tours.fr}
\academicsupervisor{Yannick}{KERGOSIEN}{yannick.kergosien@univ-tours.fr}

\resume{}

\motcle{}

\abstract{}

\keyword{}

\posterblock{}{}{images/LOGO_APHP}{}

\posterblock{}{}{images/LOGO_APHP}{}

\posterblock{}{}{images/LOGO_APHP}{}


\addbibresource{PRD2.bib}

%%%%%%%%%%%%%%%%%%%%%%%%%%%%%%%%%%%%%%%%%%%%%%%%%%%%%%%%%%%%%%%%%%%%%%%%%%%%%%%%%%%%%%%%%%%%%%%%%%%
%%%%%%%%%%%%%%%%%%%%%%%%%%%%%%%%%%%%%%%%%%%%%%%%%%%%%%%%%%%%%%%%%%%%%%%%%%%%%%%%%%%%%%%%%%%%%%%%%%%

\begin{document}

\chapter*{Introduction}

Ce rapport traite du projet Recherche et Développement au sein de l'école Polytech. Il se déroule lors la cinquième année d'étude et représente une forme de synthèse des connaissances acquises lors des années d’études précédentes. Il permet de développer et d’approfondir son savoir sur un ou plusieurs champs de compétences spécifiques, afin de devenir un spécialiste du domaine choisi dans le cadre du projet. Ce travail est mené seul, sous la supervision d'un tuteur académique et, pour certains projets, avec l’aide d’un intervenant extérieur qui fait partie des initiateurs du sujet. Il permet ainsi de développer son autonomie en menant à bien un travail des prémices jusque, dans le meilleur des cas, à la production.

J'ai décidé de me consacrer à un projet lié à un enjeu important de notre société: la gestion des patients dans un hôpital. Aujourd'hui, les hôpitaux sont l'objet de nombreux débats dans l'actualité, et l'un des points majeurs concerne la gestion des patients au sein de hôpitaux. Plusieurs problèmes peuvent être cités: un temps d'attente bien trop long pour les patients (que ce soit une fois à l'hôpital ou pour avoir un rendez-vous avec un spécialiste), des problèmes de ressource (concernant le personnel ou bien les salles à disposition) ou encore des problèmes d'ordre financier en général. Certaines de ses problématiques peuvent être étudiées selon le point de vue de l'optimisation, et la recherche opérationnelle peut permettre d'améliorer certains aspects de la gestion actuelle. Pour ma part, me pencher sur les problèmes de recherche opérationnelle en général m'intéressait, et avoir l'opportunité de réaliser un projet dans ce domaine, d'autant plus dans un secteur qui en a besoin, était une bonne occasion pour améliorer mes compétences.

Le sujet de ce travail rentre dans le cadre d’un projet visant à développer un outil de gestion temps réel de parcours de patients pour l’AP-HP (Assistance publique - Hôpitaux de Paris). L'objectif de cet outil est de gérer un ensemble de patients nécessitant plusieurs activités de soins planifiées sur une journée dans un hôpital de jour. Chaque patient doit suivre un « parcours de soins » défini par un ensemble d’activités de soins. Certaines doivent être réalisées avant d’autres (contraintes de précédences) alors que pour d'autres, l’ordre n’a pas d’importance. Il peut exister des délais d'attente à respecter entre les activités. Chaque activité de soin est caractérisée par une durée et un ensemble de ressources nécessaires pour la réaliser. Ces ressources peuvent représenter le personnel, le matériel, les salles ou encore une combinaison de ces derniers. Elles sont caractérisées par une quantité et un horaire où elles sont disponibles. L’idée de cet outil est de pouvoir changer l’horaire des activités de soins des patients en fonction des aléas (retard, patient absent, etc.) pour diminuer le temps d’attente des patients.


Plusieurs projets ont abouti à une première version simple d’un outil (plateformeweb). L’objectif de ce projet est d’ajouter quelques fonctionnalités à l'outil afin d’obtenir une version finale pour l’hôpital concerné. Ces nouvelles fonctionnalités dépendront des compétences étudiants (visualisation, RO, etc.), des priorités de l’hôpital, de nouvelles idées, etc.


\part{Analyse préliminaire et détermination des objectifs}


\chapter{Contexte de la réalisation}


\chapter{Description générale}


\chapter{État de l'art}


\chapter{Analyse et conception}


\part{Développement}


\chapter{Mise en œuvre}


\chapter{Bilan et conclusion}


%TODO Supprimer ce chapitre une fois le contenu temporaire utilisé
\chapter*{TEMPO}

Ce projet rentre dans le cadre d’un projet visant à développer un outil de gestion temps réel de parcours de patients pour l’AP-HP (Assistance publique - Hôpitaux de Paris).  L'objectif de cet outil est de gérer un ensemble de patients nécessitant plusieurs activités de soins planifiées sur une journée dans un hôpital de jour. Chaque patient doit suivre un « parcours de soins » défini par un ensemble d’activités de soins. Certaines doivent être réalisées avant d’autres (contraintes de précédences) alors que pour d'autres l’ordre n’a pas d’importance. Il peut exister des délais min et/ou max à respecter entre les activités. Chaque activité de soin est caractérisée par une durée et un ensemble de ressources nécessaires pour la réaliser. Les ressources peuvent représenter le personnel, le matériel, les salles ou encore une combinaison de ces derniers. Elles sont caractérisées par une quantité et un horaire où elles sont disponibles. L’idée de cet outil est de pouvoir changer l’horaire des activités de soins des patients en fonction des aléas (retard, patient absent, etc.) pour diminuer le temps d’attente des patients.

Plusieurs projets ont abouti à une première version simple d’un outil (plateformeweb). L’objectif de ce projet est d’ajouter quelques fonctionnalités à l'outil afin d’obtenir une version finale pour l’hôpital concerné. Ces nouvelles fonctionnalités dépendront des compétences étudiants (visualisation, RO, etc.), des priorités de l’hôpital, de nouvelles idées, etc.


\appendix

\chapter{Chiffrage du projet}

\chapter{Diagramme de Gantt}

\chapter{Manuel développeur}


\end{document}
