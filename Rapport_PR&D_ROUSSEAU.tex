\documentclass{polytech/polytech}

\typereport{prddi5}

\reportyear{2018-2019}
\title{Outil de gestion de parcours patient dans un hôpital de jour}
\student[di5]{Romain}{ROUSSEAU}{romain.rousseau@etu.univ-tours.fr}
\academicsupervisor{Yannick}{KERGOSIEN}{yannick.kergosien@univ-tours.fr}

\resume{}

\motcle{}

\abstract{}

\keyword{}

\posterblock{}{}{images/LOGO_APHP}{}

\posterblock{}{}{images/LOGO_APHP}{}

\posterblock{}{}{images/LOGO_APHP}{}


\addbibresource{PRD2.bib}

%%%%%%%%%%%%%%%%%%%%%%%%%%%%%%%%%%%%%%%%%%%%%%%%%%%%%%%%%%%%%%%%%%%%%%%%%%%%%%%%%%%%%%%%%%%%%%%%%%%
%%%%%%%%%%%%%%%%%%%%%%%%%%%%%%%%%%%%%%%%%%%%%%%%%%%%%%%%%%%%%%%%%%%%%%%%%%%%%%%%%%%%%%%%%%%%%%%%%%%

\begin{document}

\chapter*{Introduction}

Ce rapport traite du projet Recherche et Développement au sein de l'école Polytech. Il se déroule lors la cinquième année d'étude et représente une forme de synthèse des connaissances acquises lors des années d’études précédentes. Il permet de développer et d’approfondir son savoir sur un ou plusieurs champs de compétences spécifiques, afin de devenir un spécialiste du domaine choisi dans le cadre du projet. Ce travail est mené seul, sous la supervision d'un tuteur académique et, pour certains projets, avec l’aide d’un intervenant extérieur qui fait partie des initiateurs du sujet. Il permet ainsi de développer son autonomie en menant à bien un travail des prémices jusque, dans le meilleur des cas, à la production.

J'ai décidé de me consacrer à un projet lié à un enjeu important de notre société: la gestion des patients dans un hôpital. Aujourd'hui, les hôpitaux sont l'objet de nombreux débats dans l'actualité, et l'un des points majeurs concerne la gestion des patients au sein des hôpitaux. Plusieurs problèmes peuvent être cités: un temps d'attente bien trop long pour les patients (que ce soit une fois à l'hôpital ou pour avoir un rendez-vous avec un spécialiste), des problèmes de ressource (concernant le personnel ou bien les salles à disposition) ou encore des problèmes d'ordre financier en général. Certaines de ses problématiques peuvent être étudiées selon le point de vue de l'optimisation, et la recherche opérationnelle peut permettre d'améliorer certains aspects de la gestion actuelle. Pour ma part, me pencher sur les problèmes de recherche opérationnelle en général m'intéressait, et avoir l'opportunité de réaliser un projet dans ce domaine, d'autant plus dans un secteur qui en a besoin, était une bonne occasion pour améliorer mes compétences.

Le sujet de ce travail rentre dans le cadre d’un projet visant à développer un outil de gestion temps réel de parcours de patients pour l’AP-HP (Assistance publique - Hôpitaux de Paris). L'objectif de cet outil est de gérer un ensemble de patients nécessitant plusieurs activités de soins planifiées sur une journée dans un hôpital de jour. Chaque patient doit suivre un « parcours de soins » défini par un ensemble d’activités de soins. Certaines doivent être réalisées avant d'autres (contraintes de précédences) alors que pour d'autres, l’ordre n’a pas d’importance. Il peut exister des délais d'attente à respecter entre les activités. Chaque activité de soin est caractérisée par une durée et un ensemble de ressources nécessaires pour la réaliser. Ces ressources peuvent représenter le personnel, la salle où s'effectue l'activité ou encore le matériel utilisé. Elles sont caractérisées par une quantité et un horaire où elles sont disponibles. L'idée de l'outil est de pouvoir proposer un calendrier adaptatif dans lequel les horaires des activités de soins des patients peuvent être ajustés en fonction des aléas pouvant apparaître (retard, patient absent, etc.) afin de diminuer du mieux possible le temps d'attente des patients.

Ce projet est la suite de plusieurs travaux effectués à l'école ces dernières années et qui ont abouti à une première version simple d'un outil sous forme de plateforme web. L'objectif est d'améliorer l'outil actuel en corrigeant certaines problèmes et en ajoutant des fonctionnalités afin de se rapprocher d'une version finale exploitable à l'avenir. Parmi les fonctionnalités qui seront ajoutées, on peut notamment penser à la planification automatique des activités de soin sur le calendrier ou encore l'ajout de notions de temps réel sur les éléments du calendrier.

La première partie de ce rapport sera consacrée à l'analyse préliminaire de la plateforme et à la détermination des tâches à réaliser. Elle comprend des éléments sur le contexte de la réalisation, la description de la plateforme telle qu'elle se trouvait au début du projet, un état de l'art sur les outils existant et sur la gestion de la planification des activités, et la présentation des tâches qui seront effectuées durant les semaines de projet. La seconde partie sera dédiée aux détails des développements de la plateforme. Des documents complémentaires sont disponibles en annexe de ce rapport comme les diagrammes de Gantt, les diagrammes UML, des éléments de gestion de projet, etc. Un guide développeur destiné aux personnes qui récupèreront le projet est également présent en complément.


\part{Analyse préliminaire et détermination des objectifs}


\chapter{Contexte de la réalisation}


\chapter{Description générale}


\chapter{État de l'art}


\chapter{Analyse et conception}


\part{Développement}


\chapter{Mise en œuvre}


\chapter{Bilan et conclusion}


\appendix

\chapter{Chiffrage du projet}

\chapter{Diagramme de Gantt}

\chapter{Manuel développeur}


\end{document}
