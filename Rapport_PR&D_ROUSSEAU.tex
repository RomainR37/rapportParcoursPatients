\documentclass{polytech/polytech}

\typereport{prddi5}

\reportyear{2018-2019}
\title{Outil de gestion de parcours patient dans un hôpital de jour}
\student[di5]{Romain}{ROUSSEAU}{romain.rousseau@etu.univ-tours.fr}
\academicsupervisor{Yannick}{KERGOSIEN}{yannick.kergosien@univ-tours.fr}

\resume{}

\motcle{}

\abstract{}

\keyword{}

\posterblock{}{}{images/LOGO_APHP}{}

\posterblock{}{}{images/LOGO_APHP}{}

\posterblock{}{}{images/LOGO_APHP}{}


%%%%%%%%%%%%%%%%%%%%%%%%%%%%%%%%%%%%%%%%%%%%%%%%%%%%%%%%%%%%%%%%%%%%%%%%%%%%%%%%%%%%%%%%%%%%%%%%%%%
%%%%%%%%%%%%%%%%%%%%%%%%%%%%%%%%%%%%%%%%%%%%%%%%%%%%%%%%%%%%%%%%%%%%%%%%%%%%%%%%%%%%%%%%%%%%%%%%%%%

\begin{document}

\chapter*{Introduction}

%TODO Supprimer ce chapitre une fois le contenu temporaire utilisé
\chapter*{TEMPO}

Ce projet rentre dans le cadre d’un projet visant à développer un outil de gestion temps réel de parcours de patients pour l’AP-HP (Assistance publique - Hôpitaux de Paris).  L’objectif de cet outil est de gérer un ensemble de patients nécessitant plusieurs activités de soins planifiées sur une journée dans un hôpital de jour. Chaque patient doit suivre un « parcours de soins » défini par un ensemble d’activités de soins. Certaines doivent être réalisées avant d’autres (contraintes de précédences) alors que pour d’autres l’ordre n’a pas d’importance. Il peut exister des délais min et/ou max à respecter entre les activités. Chaque activité de soin est caractérisée par une durée et un ensemble de ressources nécessaires pour la réaliser. Les ressources peuvent représenter le personnel, le matériel, les salles ou encore une combinaison de ces derniers. Elles sont caractérisées par une quantité et un horaire où elles sont disponibles. L’idée de cet outil est de pouvoir changer l’horaire des activités de soins des patients en fonction des aléas (retard, patient absent, etc.) pour diminuer le temps d’attente des patients.

Plusieurs projets ont abouti à une première version simple d’un outil (plateformeweb). L’objectif de ce projet est d’ajouter quelques fonctionnalités à l’outil afin d’obtenir une version finale pour l’hôpital concerné. Ces nouvelles fonctionnalités dépendront des compétences étudiants (visualisation, RO, etc.), des priorités de l’hôpital, de nouvelles idées, etc.


\end{document}
