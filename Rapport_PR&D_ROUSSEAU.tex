\documentclass{polytech/polytech}

\typereport{prddi5}

\reportyear{2018-2019}
\title{Outil de gestion de parcours patient dans un hôpital de jour}
\student[di5]{Romain}{ROUSSEAU}{romain.rousseau@etu.univ-tours.fr}
\academicsupervisor{Yannick}{KERGOSIEN}{yannick.kergosien@univ-tours.fr}

\resume{}

\motcle{}

\abstract{}

\keyword{}

\posterblock{}{}{images/LOGO_APHP}{}

\posterblock{}{}{images/LOGO_APHP}{}

\posterblock{}{}{images/LOGO_APHP}{}


\addbibresource{PRD2.bib}

%%%%%%%%%%%%%%%%%%%%%%%%%%%%%%%%%%%%%%%%%%%%%%%%%%%%%%%%%%%%%%%%%%%%%%%%%%%%%%%%%%%%%%%%%%%%%%%%%%%
%%%%%%%%%%%%%%%%%%%%%%%%%%%%%%%%%%%%%%%%%%%%%%%%%%%%%%%%%%%%%%%%%%%%%%%%%%%%%%%%%%%%%%%%%%%%%%%%%%%

\begin{document}

\chapter*{Introduction}

Ce rapport traite du projet Recherche et Développement au sein de l'école Polytech. Il se déroule lors la cinquième année d'étude et représente une forme de synthèse des connaissances acquises lors des années d’études précédentes. Il permet de développer et d’approfondir son savoir sur un ou plusieurs champs de compétences spécifiques, afin de devenir un spécialiste du domaine choisi dans le cadre du projet. Ce travail est mené seul, sous la supervision d'un tuteur académique et, pour certains projets, avec l’aide d’un intervenant extérieur qui fait partie des initiateurs du sujet. Il permet ainsi de développer son autonomie en menant à bien un travail des prémices jusque, dans le meilleur des cas, à la production.

J'ai décidé de me consacrer à un projet lié à un enjeu important de notre société: la gestion des patients dans un hôpital. Aujourd'hui, les hôpitaux sont l'objet de nombreux débats dans l'actualité, et l'un des points majeurs concerne la gestion des patients au sein des hôpitaux. Plusieurs problèmes peuvent être cités: un temps d'attente bien trop long pour les patients (que ce soit une fois à l'hôpital ou pour avoir un rendez-vous avec un spécialiste), des problèmes de ressource (concernant le personnel ou bien les salles à disposition) ou encore des problèmes d'ordre financier en général. Certaines de ses problématiques peuvent être étudiées selon le point de vue de l'optimisation, et la recherche opérationnelle peut permettre d'améliorer certains aspects de la gestion actuelle. Pour ma part, me pencher sur les problèmes de recherche opérationnelle en général m'intéressait, et avoir l'opportunité de réaliser un projet dans ce domaine, d'autant plus dans un secteur qui en a besoin, était une bonne occasion pour améliorer mes compétences.

Le sujet de ce travail rentre dans le cadre d’un projet visant à développer un outil de gestion temps réel de parcours de patients pour l’AP-HP (Assistance publique - Hôpitaux de Paris). L'objectif de cet outil est de gérer un ensemble de patients nécessitant plusieurs activités de soins planifiées sur une journée dans un hôpital de jour. Chaque patient doit suivre un « parcours de soins » défini par un ensemble d’activités de soins. Certaines doivent être réalisées avant d'autres (contraintes de précédences) alors que pour d'autres, l’ordre n’a pas d’importance. Il peut exister des délais d'attente à respecter entre les activités. Chaque activité de soin est caractérisée par une durée et un ensemble de ressources nécessaires pour la réaliser. Ces ressources peuvent représenter le personnel, la salle où s'effectue l'activité ou encore le matériel utilisé. Elles sont caractérisées par une quantité et un horaire où elles sont disponibles. L'idée de l'outil est de pouvoir proposer un calendrier adaptatif dans lequel les horaires des activités de soins des patients peuvent être ajustés en fonction des aléas pouvant apparaître (retard, patient absent, etc.) afin de diminuer du mieux possible le temps d'attente des patients.

Ce projet est la suite de plusieurs travaux effectués à l'école ces dernières années et qui ont abouti à une première version simple d'un outil sous forme de plateforme web. L'objectif est d'améliorer l'outil actuel en corrigeant certaines problèmes et en ajoutant des fonctionnalités afin de se rapprocher d'une version finale exploitable à l'avenir. Parmi les fonctionnalités qui seront ajoutées, on peut notamment penser à la planification automatique des activités de soin sur le calendrier ou encore l'ajout de notions de temps réel sur les éléments du calendrier.

La première partie de ce rapport sera consacrée à l'analyse préliminaire de la plateforme et à la détermination des tâches à réaliser. Elle comprend des éléments sur le contexte de la réalisation, la description de la plateforme telle qu'elle se trouvait au début du projet, un état de l'art sur les outils existant et sur la gestion de la planification des activités, et la présentation des tâches qui seront effectuées durant les semaines de projet. La seconde partie sera dédiée aux détails des développements de la plateforme. Des documents complémentaires sont disponibles en annexe de ce rapport comme les diagrammes de Gantt, les diagrammes UML, des éléments de gestion de projet, etc. Un guide développeur destiné aux personnes qui désireront reprendre le projet à l'avenir est également présent en complément.


\part{Analyse préliminaire et détermination des objectifs}


Cette partie correspond à la première phase du Projet Recherche et Développement, la partie Recherche. Dans le cadre de ce projet, cette partie sera une analyse générale de la plateforme existante afin de déterminer quels objectifs nous devrons atteindre à l'issue de ce projet. Le premier chapitre sera consacré au contexte de la réalisation, la raison de l'initiation du projet ainsi que les acteurs concernés. Le chapitre suivant décrira la plateforme telle qu'elle se trouvait à la reprise du sujet, en détaillant la structure et les fonctionnalités qu'elle contient. Enfin le dernier chapitre évoquera des éléments de recherche opérationnelle dans le domaine de la gestion des patients dans un hôpital. 


\chapter{Contexte de la réalisation et objectifs}

Ce chapitre présente le contexte du sujet et les objectifs à atteindre à l'issue du projet.

\section{Contexte, enjeux et acteurs}


Le secteur de la santé est un sujet prédominant dans la société actuelle. De nombreux débats existent concernant la modernisation des structures hospitalières et surtout, la réduction des dépenses dans le secteur. Cependant, ces problématiques sont très complexes à gérer car il est important de maintenir une qualité d'accueil nécessaire au bon traitement des patients, néanmoins une baisse des dépenses pourrait se répercuter dans la qualité des prestations. L'un des aspects sur lesquels ce projet sera consacré est la gestion de prise de rendez-vous médicale. Les consultations sont soumises à de nombreux aléas, que ce soit côté patient comme côté médecin. Il n'est pas rare pour un patient de devoir attendre plusieurs mois avant d'avoir un rendez-vous chez un spécialiste, tout comme il est fréquent pour un médecin de voir un patient absent lors d'un rendez-vous. Ceci entraîne de la frustration et du gaspillage de ressources en général. Des solutions dans le domaine de la recherche opérationnelle peuvent être mises en place, afin d'améliorer la gestion des ressources médicales et de réduire les temps d'attente pour les patients, que ce soit avant ou pendant la prise en charge. 

Les services hospitaliers sont souvent surchargés par le nombre de patients dans l'établissement. Il est de plus en plus fréquent de voir les hôpitaux proposer des services en ambulatoire, c'est-à-dire prendre en charge un patient sur une journée, évitant ainsi les frais pouvant s'appliquer lorsqu'un patient dort sur place. Les dépenses liées aux hospitalisations sont parmi les plus importantes dans un établissement et de nos jours, les hôpitaux proposent même des alternatives pour que les patients ne restent pas sur place la nuit, comme par exemple des chambres d'hôtel afin de ne pas occuper une chambre pendant l'hospitalisation \cite{noauthor_chu_nodate}. Un établissement, ou une partie d'établissement, proposant ce type de service est appelé "hôpital de jour".

L'idée d'implémentation d'un outil numérique dans une structure implique une réflexion sur les concepts liés l'hospitalisation d'un patient. Il est possible de définir, selon les pathologies à traiter, un parcours-patient à suivre. Un parcours-patient (ou parcours-clinique) est un plan de soins pluridisciplinaires exposant les étapes que doivent un patient pour un problème spécifique. On peut les décomposer en plusieurs activités prédéfinies, certaines devant être réalisées avant d'autres (ce qu'on appelle une contrainte de précédence), nécessitant des ressources physiques ou matérielles (contrainte de ressources) et devant être effectuées dans un délai ou un créneau horaire précis (contrainte temporelles). La mise en place d'un système de centralisation des parcours-patient reste difficile, notamment car elle repose sur une bonne communication entre les services d'un établissement. Néanmoins, cela permettrait d'avoir potentiellement d'avoir une meilleure utilisation des ressources de l'hôpital et par conséquent, un effet favorable sur les durées d'hospitalisation et les dépenses hospitalières entre autres.

L'Assistance Publique - Hôpitaux de Paris (AP-HP), dans son plan stratégique 2015 - 2019, a lancé une réflexion sur la mise en place d'une clinique ambulatoire. Avec la mutualisation de certains services au sein de différents hôpitaux de la région, le site de l'hôpital \textit{Béclère}, à Clamart dans le département des Hauts-de-Seine, obtiendra un gain de place suffisant pour proposer une telle solution. Il regroupera tous les hôpitaux de jour existants du site, avec un management centralisé et transversal à l'aide de parcours-patients. Les équipes médicales et administratives ont pris le soin de définir d'ors et déjà une vingtaine de parcours patients en adéquation avec les ressources déjà en place, et présenté une répartition type des prises en charge hebdomadaire. Celle-ci devrait en théorie permettre d'accueillir 179 patients chaque semaine, et à terme environ 1000 par mois. Ainsi, la conception d'un système d'aide à la décision pour la prise de rendez-vous serait utile pour une nouvelle structure comme celle-ci.

Le sujet a été proposé en 2015 par Lucie ROUSSEL, ingénieur en ordonnancement à l'AP-HP et a été suivi lors de plusieurs projets à Polytech: tout d'abord, lors d'un projet collectif (réalisé par Benjamin COLMART, Jean COQUELET, Anaëlle HAMON, Jiang MING, Yan LI et Minghui ZHANG) qui a posé les bases de la modélisation et des premières fonctionnalités, un projet R\&D par Jean COQUELET en 2016 également basé davantage sur l'aspect gestion et optimisation des parcours, un projet R\&D par Guillaume POCHET en 2017 qui a ajouté des éléments sur la gestion de calendrier et enfin un autre projet R\&D réalisé par Jing YANG qui ajoute des améliorations générales. Le projet tel qu'il est développé à Polytech servira dans un premier temps en tant que \textit{proof of concept}, et si celui-ci satisfait l'administration, il pourra entrer en production à plus grande échelle dans le futur.

La plateforme, une fois installée sur mon poste à la reprise du projet, fonctionne. Il est possible d'ajouter des patients et de leur associer un parcours-patient avec des heures de rendez-vous et des activités à suivre lors d'une journée. Cependant, de nombreux éléments sont à améliorer, des bugs à corriger dans un premier temps mais aussi des améliorations d'un point de vue graphique. Des fonctionnalités importantes restent à implémenter: d'un côté, l'ajout d'une notion de temps réel sur les activités et les rendez-vous, et de l'autre, la possibilité de faire une planification automatique tenant compte des patients et des ressources disponibles. Compte-tenu du temps alloué pour le projet, il sera difficile d'ajouter ces deux fonctionnalités importantes. Nous verrons par la suite vers quels éléments nous allons nous pencher en priorité. 


\section{Objectifs}


Suite à l'affectation du sujet, les premières rencontres avec M. KERGOSIEN ont permis de fixer des objectifs principaux pour mener à bien le projet. Trois axes de développement ont été définis pour la finalisation de la plateforme : la correction de certains problèmes sur les fonctionnalités existantes, l'ajout de fonctions de planification automatique et l'implémentation de fonctionnalités permettant de planifier en temps réel. 

Les trois objectifs présentés par la suite sont conséquents en terme de temps. Il sera sans doute très compliquer de réaliser les trois objectifs dans la limite de temps alloué au projet. Nous nous focaliserons ainsi sur deux des trois objectifs à suivre. 


\subsection{Corrections des bugs et améliorations sur les fonctionnalités existantes}

Le premier objectif consiste à corriger certaines erreurs que l'on peut trouver sur la plateforme. À la première analyse du projet, plusieurs problèmes sont apparus, que ce soit des erreurs d'affichage, des soucis d'ergonomie par exemple. Ainsi, le but est d'améliorer les éléments déjà existants de la plateforme afin de la rendre plus stable et de meilleur qualité. Parmi les corrections et améliorations à apporter, on retrouve la révision de la suppression de données, de l'onglet "Plan de parcours", du formulaire de création de patient, de certains éléments de la page d'accueil. 

Plusieurs fonctionnalités qui n'ont pas encore été implémentées sont aussi à rajouter pour enrichir les possibilités offertes à l'utilisateur, comme l'affichage du planning pour un patient donné, l'ajout de fonctions de tri pour l'affichage des ressources, ou encore, la création de jeux de données permettant de tester plus facilement la planification. 


\subsection{Implémentation de la planification automatique}


\subsection{Implémentation de la planification temps réel}



\section{Bases méthodologiques}


Gestion de projet


La plateforme est déjà en MVC.


\chapter{Description générale}


\section{Environnement du projet}


%% PRD Jing 
Ce projet est la reprise de trois projets existants, un projet de la gestion commencé par un groupe de 6 étudiants lors de leur 5ème année à l'école polytechnique de Tours de l'année 2015-2016 dont Jean Coquelet fait chef de projet, un projet de l'optimisation réalisé aussi par Jean Coquelet dans le cadre de son PRD de l'année 2015-2016 et un projet de développement fait par Pochet Guillaume de son PRD de l'année 2016-2017.

L'application existante est exécutée sur un serveur qui communiquera avec une base de données pour pouvoir récupérer les différentes informations de l'application. Un ordinateur ou smartphone (pour les patients) avec un navigateur sera demandé pour l'implémentation de cette application. Voici le diagramme de déploiement Figure 1. 

Cette application existante a réalisé des fonctionnalités pour la gestion des ressources, des patients et des parcours, elle nous permet de planifier des parcours aux patients et de visualiser le planning d'un patient ou d'un personnel. Les fonctionnalités détaillées seront décrites dans la partie fonctionnalités déjà implémentées (voir Annexes).

Les pages webs avec des interfaces existent déjà. Ces interfaces utilisent tous les modules définis dans le diagramme de classe Figure 1(Annexe D). Dans mon PRD, je vais rajouter des éléments basés sur les interfaces existantes et modifier ce que n’a pas été bien développé. 

Je vais revoir la suppression des éléments qui n'est pas bien résolu pendant l'ancien projet en restructurant une partie de la base de données. Pour la gestion de patient, je vais rajouter de nouvelles fonctionnalités et modifier les actions sur un patient cherché sur l'interface « gérer patient ». À part de ça, l'interface suivante qui est « dossier parcours » sera modifiée et les onglets et champs seront ajoutés ici. Selon l'interface « nouveau parcours », je vais retravailler sur la partie disponibilité qui ne changera pas l'interface.


\section{Caractéristiques des utilisateurs}


%% PRD Jing

Dans notre projet, on peut différencier 3 types d’utilisateurs de l’application web :

\begin{itemize}
	\item Les Patients : Accès limité à l'application, ils pourront uniquement visualiser leur planning et suivre l'avancement dans leur parcours (sans indications horaires excepté sur la prochaine activité).
	\item Le personnel de soins : Ils auront accès à l'ajout / la recherche de patients, l'accès à leur planning et à la visualisation d'un planning d'un patient.
	\item Administrateurs et infirmières de coordination : C'est l'utilisateur qui a tous les droits sur l'application. Il peut visualiser/modifier les plannings de toutes les ressources, ajouter / modifier une activité, gérer les ressources (humaines ou matérielles), ajouter/modifier un parcours patient.
\end{itemize}

Les différents utilisateurs de l'application devront s'authentifier pour pouvoir accéder aux
différentes fonctionnalités.


\section{Fonctionnalités du système}


Les fonctionnalités existantes sont dispo en annexe.



\section{Structure générale du système}


\chapter{État de l'art}


\section{Planification automatique}


\part{Développement}


\chapter{Analyse et conception}


\chapter{Mise en œuvre}


\chapter{Bilan et conclusion}


\appendix

\chapter{Fonctionnalités existantes}



\chapter{Explications des tables du modèle}


%%PRD Jing

Voici l'explication sur les tables créées dans l'ancien projet [2]que je vais continuer à utiliser.

\begin{description}
	\item[Activité] : Table regroupant l'ensemble des informations concernant une activité. C'est cet élément qui constitue les parcours. Une activité peut obliger la réalisation de certaines autres activités avant d'être réalisable. Cette notion de dépendance dépend du parcours en cours. Cette notion de précédence est sera vue plus loin via la table composer. En plus de cela, une activité a besoin de ressources. Ce lien se fait par la table nécessiter.
	\item[Champ] : Table contenant l'ensemble des champs qu'il est possible d'ajouter dans un onglet d'un dossier Parcours. À chaque champ est lié un type de champ, que nous verrons plus loin, avec la table typechamp.
	\item[Composer] : Table permettant de faire le lien entre un parcours et une activité. Par l'intermédiaire de cette table nous pouvons dire en fonction d'un parcours et d'une activité s'il y a des besoins en termes de précédence. Chaque ligne, dans cette table, a pour signification : « Pour l'activité A dans le parcours P, il faut avoir réalisé l'activité B avant, et ce dans un délai compris entre delaiMin et delaiMax. ». Il est important de noter qu'il est possible de mettre à « null », l'id de l'activité précédente.
	\item[Compte] : Table regroupant l'ensemble des comptes qu'ils soient des comptes patient ou des comptes de type ressource médicale.
	\item[Constituerdossier] : Table permettant de faire le lien entre un dossier parcours et les informations qui le constituent. En effet, nous retrouvons pour chaque dossier parcours et pour chaque onglet dans ce dernier, la valeur des champs le composant.
	\item[Dossiergenerique] : Table définissant, pour un parcours, les onglets et les champs que tous les dossiers parcours doivent avoir impérativement.
	\item[Dossierparcours] : Table renseignant les informations génériques d'un dossier parcours. Soit le patient associé, le parcours, dates de création et de dernière modification.
	\item[Etreindisponible] : Table regroupant l'ensemble des indisponibilités pour une ressource. Cette indisponibilité est caractérisée par une date de début et de fin, acceptant toutes les deux le renseignement de l'heure.
	\item[Jour] : Table qui contient les jours de semaine, ainsi que leurs index sous MySQL. Cette table a un enjeu au niveau des prévisions du nombre de patient par jour pour un parcours.
	\item[Necessiter] : Table permettant de renseigner les types de ressources requises pour une activité, ainsi que la quantité nécessaire.
	\item[Onglet] : Table des onglets disponibles pour constituer un dossier parcours.
	\item[Parcours]	 : Table décrivant un parcours de façon générale.
	\item[Personnel] : Table contenant l'ensemble de personnel médical de l'établissement. Chaque personnel a un compte, et est considéré comme une ressource.
	\item[Planparcours] : Table regroupant l'ensemble des objectifs concernant le nombre de patients pour un parcours pour un jour donné.
	\item[Ressource] : Table faisant le lien entre la table typeressource et personnel ou matériel. Ce lien	sera expliqué plus en détail avec la table typeressource.
	\item[Typechamp] : Table regroupant les différents type de champs qu'il est possible d'ajouter dans un dossier parcours. Elle contient également le code HTML des composants, permettant ainsi une	mise en page en accord avec les autres éléments des pages.
	\item[Typecompte] : Table utilisée pour la gestion des droits.
	\item[Typeressource] : Table contenant les types de ressource d'un point de vue activité. En effet, une activité peut avoir besoin d'un type de ressource bien caractéristique (ex : IDE obésité). C'est pourquoi nous avons un double niveau de type de ressource. Un concernant les activités (typeressource) et un second plus d’un point de vue logique générale (personnel, matériel).
	\item[Ordonnancer] : Table de fait de notre système. C'est la table la plus importante. Chaque ligne veut dire : « Pour le patient P qui fait le parcours Pa à la date D, il a besoin de la ressource R pour faire l’activité A de start à end. Cette table contient la planification réalisée de manière manuelle ou automatique (pas implémenté pour le moment).
	\item[Evènement] : Table identique à la table Ordonnancer. Elle a le même but que la table ordonnancer mais cette table contient la planification en cours. Cette table permet de donc de pouvoir sauvegarder ou restaurer la planification en fonction des modifications que l’utilisateur a effectuées.
\end{description}


\chapter{Chiffrage du projet}

Revoir la suppression : 1 j
Revoir l'onglet "Plan de parcours" : 1,5 j
Revoir le formulaire de création de patient : 2 j
Revoir la page d'accueil : 1 j
Affichage du planning patient : 5 j
Ajouter des tris dans les onglets d'affichage : 1 j
Corriger le problème des URL : 2 j
Affichage des ressources sélectionnées et test planning : 10 j
Planification automatique : 20 j
Jeux de données : 3 j
Rapport : 15 j
Préparation soutenance : 3 j

\chapter{Diagramme de Gantt}

\chapter{Gestion de projet}

\chapter{Manuel développeur}


%TODO Supprimer le chapitre une fois terminé
\chapter{TEMPO}

\section*{Suppression}

La fonctionnalité "Supprimer" a été ajoutée lors du dernier PR\&D réalisé par Yang Jing. Elle est implémentée dans les onglets de Gestion de la plateforme pour avoir la possibilité de supprimer les ressources qui ne sont pas nécessaires ou qui sont devenues obsolètes.  

Cette fonctionnalité a des risques importants. En effet, la suppression d'un élément peut entraîner des répercussions sur l'intégralité de la plateforme (exemple ...). 

À l'heure actuelle, les boutons "Supprimer" ne disposent d'aucune sécurité, ce qui peut se révéler très dangereux. Un simple clic sur le bouton entraîne la disparition de la ressource sélectionnée. Compte tenu de la dangerosité de la manœuvre sur le bon fonctionnement de la plateforme, il est nécessaire de corriger ce problème. 

À minima, l'apparition d'un message de confirmation lors du clic sur le bouton permettrait d'éviter les mauvaises manipulations. 

Pour afficher un message de confirmation, il suffit d'ajouter \javacode{onclick="return confirm('Etes vous sur ?');"} sur le lien de l'action ...


\end{document}
